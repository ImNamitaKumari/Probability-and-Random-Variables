\documentclass[journal,12pt,twocolumn]{IEEEtran}
%
\usepackage{setspace}
\usepackage{gensymb}
\usepackage{caption}
%\doublespacing
\singlespacing
%\usepackage{graphicx}
%\usepackage{amssymb}
%\usepackage{relsize}
\usepackage[cmex10]{amsmath}
%\usepackage{amsthm}
%\interdisplaylinepenalty=2500
%\savesymbol{iint}
%\usepackage{txfonts}
%\restoresymbol{TXF}{iint}
%\usepackage{wasysym}
\usepackage{amsthm}
%\usepackage{iithtlc}
\usepackage{mathrsfs}
\usepackage{txfonts}
\usepackage{stfloats}
\usepackage{bm}
\usepackage{cite}
\usepackage{cases}
\usepackage{subfig}
\usepackage{xtab}
\usepackage{multirow}
%\usepackage{algorithm}
%\usepackage{algpseudocode}
\usepackage{booktabs}
\usepackage{enumitem}
\usepackage{mathtools}
\usepackage{tikz}
\usepackage{pgfplots}
\usepackage{circuitikz}
\usepackage{verbatim}
\usepackage{tfrupee}
\usepackage[breaklinks=true]{hyperref}
%\usepackage{stmaryrd}
\usepackage{tkz-euclide} % loads  TikZ and tkz-base
%\usetkzobj{all}
\usetikzlibrary{fit}
\usetikzlibrary{calc,math}
%\pgfdeclarelayer{background}
%\pgfsetlayers{background}
\usepackage{listings}
    \usepackage{color}                                            %%
    \usepackage{array}                                            %%
    \usepackage{longtable}                                        %%
    \usepackage{calc}                                             %%
    \usepackage{multirow}                                         %%
    \usepackage{hhline}                                           %%
    \usepackage{ifthen}                                           %%
  %optionally (for landscape tables embedded in another document): %%
    \usepackage{lscape}     
\usepackage{multicol}
\usepackage{chngcntr}
%\usepackage{enumerate}

%\usepackage{wasysym}
%\newcounter{MYtempeqncnt}
\DeclareMathOperator*{\Res}{Res}
%\renewcommand{\baselinestretch}{2}
\renewcommand\thesection{\arabic{section}}
\renewcommand\thesubsection{\thesection.\arabic{subsection}}
\renewcommand\thesubsubsection{\thesubsection.\arabic{subsubsection}}

\renewcommand\thesectiondis{\arabic{section}}
\renewcommand\thesubsectiondis{\thesectiondis.\arabic{subsection}}
\renewcommand\thesubsubsectiondis{\thesubsectiondis.\arabic{subsubsection}}

% correct bad hyphenation here
\hyphenation{op-tical net-works semi-conduc-tor}
\def\inputGnumericTable{}                                 %%

\lstset{
%language=C,
frame=single, 
breaklines=true,
columns=fullflexible
}
%\lstset{
%language=tex,
%frame=single, 
%breaklines=true
%}

\begin{document}
%


\newtheorem{theorem}{Theorem}[section]
\newtheorem{problem}{Problem}
\newtheorem{proposition}{Proposition}[section]
\newtheorem{lemma}{Lemma}[section]
\newtheorem{corollary}[theorem]{Corollary}
\newtheorem{example}{Example}[section]
\newtheorem{definition}[problem]{Definition}
%\newtheorem{thm}{Theorem}[section] 
%\newtheorem{defn}[thm]{Definition}
%\newtheorem{algorithm}{Algorithm}[section]
%\newtheorem{cor}{Corollary}
\newcommand{\BEQA}{\begin{eqnarray}}
\newcommand{\EEQA}{\end{eqnarray}}
\newcommand{\define}{\stackrel{\triangle}{=}}
\newcommand{\RomanNumeralCaps}[1]
    {\MakeUppercase{\romannumeral #1}}
\bibliographystyle{IEEEtran}
%\bibliographystyle{ieeetr}
\providecommand{\mbf}{\mathbf}
\setlength{\parskip}{1em}
\providecommand{\pr}[1]{\ensuremath{\Pr\left(#1\right)}}
\providecommand{\qfunc}[1]{\ensuremath{Q\left(#1\right)}}
\providecommand{\sbrak}[1]{\ensuremath{{}\left[#1\right]}}
\providecommand{\lsbrak}[1]{\ensuremath{{}\left[#1\right.}}
\providecommand{\rsbrak}[1]{\ensuremath{{}\left.#1\right]}}
\providecommand{\brak}[1]{\ensuremath{\left(#1\right)}}
\providecommand{\lbrak}[1]{\ensuremath{\left(#1\right.}}
\providecommand{\rbrak}[1]{\ensuremath{\left.#1\right)}}
\providecommand{\cbrak}[1]{\ensuremath{\left\{#1\right\}}}
\providecommand{\lcbrak}[1]{\ensuremath{\left\{#1\right.}}
\providecommand{\rcbrak}[1]{\ensuremath{\left.#1\right\}}}
\theoremstyle{remark}
\newtheorem{rem}{Remark}
\newcommand{\sgn}{\mathop{\mathrm{sgn}}}
\providecommand{\res}[1]{\Res\displaylimits_{#1}} 
%\providecommand{\norm}[1]{\lVert#1\rVert}
\providecommand{\mtx}[1]{\mathbf{#1}}
\providecommand{\fourier}{\overset{\mathcal{F}}{ \rightleftharpoons}}
%\providecommand{\hilbert}{\overset{\mathcal{H}}{ \rightleftharpoons}}
\providecommand{\system}{\overset{\mathcal{H}}{ \longleftrightarrow}}
	%\newcommand{\solution}[2]{\textbf{Solution:}{#1}}
\newcommand{\solution}{\noindent \textbf{Solution: }}
\newcommand{\cosec}{\,\text{cosec}\,}
\providecommand{\dec}[2]{\ensuremath{\overset{#1}{\underset{#2}{\gtrless}}}}
\newcommand{\myvec}[1]{\ensuremath{\begin{pmatrix}#1\end{pmatrix}}}
\newcommand{\mydet}[1]{\ensuremath{\begin{vmatrix}#1\end{vmatrix}}}
\newcommand*{\permcomb}[4][0mu]{{{}^{#3}\mkern#1#2_{#4}}}
\newcommand*{\perm}[1][-3mu]{\permcomb[#1]{P}}
\newcommand*{\comb}[1][-1mu]{\permcomb[#1]{C}}
%\newcommand*{\perm}[2]{{}^{#1}\!P_{#2}}%
%\newcommand*{\comb}[2]{{}^{#1}C_{#2}}%
%\numberwithin{equation}{section}
\numberwithin{equation}{subsection}
%\numberwithin{problem}{section}
%\numberwithin{definition}{section}
\makeatletter
\@addtoreset{figure}{problem}
\makeatother
\let\StandardTheFigure\thefigure
\let\vec\mathbf
%\renewcommand{\thefigure}{\theproblem.\arabic{figure}}
\renewcommand{\thefigure}{\theproblem}
%\setlist[enumerate,1]{before=\renewcommand\theequation{\theenumi.\arabic{equation}}
%\counterwithin{equation}{enumi}
%\renewcommand{\theequation}{\arabic{subsection}.\arabic{equation}}
\def\putbox#1#2#3{\makebox[0in][l]{\makebox[#1][l]{}\raisebox{\baselineskip}[0in][0in]{\raisebox{#2}[0in][0in]{#3}}}}
     \def\rightbox#1{\makebox[0in][r]{#1}}
     \def\centbox#1{\makebox[0in]{#1}}
     \def\topbox#1{\raisebox{-\baselineskip}[0in][0in]{#1}}
     \def\midbox#1{\raisebox{-0.5\baselineskip}[0in][0in]{#1}}
\vspace{3cm}
\title{Assignment 4}
\author{Namita Kumari - CS20BTECH11034}
\maketitle
\newpage
\bigskip
Download all latex-tikz codes from 
%
\begin{lstlisting}
https://github.com/ImNamitaKumari/Probability-and-Random-Variables/blob/main/Assignment4/Assignment4.tex
\end{lstlisting}
\section{CSIR UGC NET - June 2015 Q. 105}
Suppose $X_1, X_2,$ ... are independent random variables. Assume that $X_1,X_3,$ ... are identically distributed with mean $\mu_1$ and variance $\sigma_1^2$, while $X_2,X_4,$ ... are identically distributed with mean $\mu_2$ and variance $\sigma_2^2$. Let $S_n=X_1+X_2+...+X_n$. Then $\frac{S_n-a_n}{b_n}$ converges in distribution to $N(0,1)$ if
\begin{enumerate}
    \item $a_n=\frac{n(\mu_1+\mu_2)}{2}$ and $b_n=\sqrt{n}\sqrt{\frac{\sigma_1^2+\sigma_2^2}{2}}$
    \item $a_n=\frac{n(\mu_1+\mu_2)}{2}$ and $b_n=\frac{n(\sigma_1+\sigma_2)}{2}$
    \item $a_n=n(\mu_1+\mu_2)$ and $b_n=\sqrt{n}\frac{(\sigma_1+\sigma_2)}{2}$
    \item $a_n=n(\mu_1+\mu_2)$ and $b_n=\sqrt{n}\sqrt{\frac{\sigma_1^2+\sigma_2^2}{2}}$
\end{enumerate}
\section{Solution}
\begin{lemma}
\label{3lemma}
For X and Y being independent random variables,
\begin{align}
    E(X+Y)=E(X)+E(Y)\label{1}\\
    Var(X+Y)=Var(X)+Var(Y)\label{3}\\
    Var(aX+b)=a^2Var(X)\label{2}
    \end{align}
    \end{lemma}
\begin{corollary}
\label{eq2.0.4}
\begin{align}
E(S_n)=
    \begin{cases}
    \frac{n(\mu_1+\mu_2)}{2} & n=\mathrm{even}\\
    \frac{n(\mu_1+\mu_2)}{2}+\frac{\mu_1-\mu_2}{2} & n=\mathrm{odd}
    \end{cases}
\end{align}
\end{corollary}
\begin{proof}
\begin{align}
    S_n=X_1+X_2+...+X_n
\end{align}
Using lemma \eqref{3lemma},
\begin{align}
    \implies E(S_n)=E(X_1)+E(X_2)+...+E(X_n)\\
    \implies E(S_n)=
    \begin{cases}
    \frac{n(\mu_1+\mu_2)}{2} & n=\mathrm{even}\\
    \frac{n(\mu_1+\mu_2)}{2}+\frac{\mu_1-\mu_2}{2} & n=\mathrm{odd}
    \end{cases}
\end{align}
\end{proof}

\begin{corollary}
\begin{align}
Var(S_n)=
    \begin{cases}
    \frac{n(\sigma_1^2+\sigma_2^2)}{2} & n=\mathrm{even}\\
    \frac{n(\sigma_1^2+\sigma_2^2)}{2}+\frac{\sigma_1^2-\sigma_2^2}{2} & n=\mathrm{odd}
    \end{cases}
\end{align}
\end{corollary}
\begin{proof}
\begin{align}
S_n=X_1+X_2+...+X_n
\end{align}
Using lemma \eqref{3lemma}
\begin{align}
    Var(S_n)=Var(X_1)+Var(X_2)+...+Var(X_n)\\
    \implies Var(S_n)=
    \begin{cases}
    \frac{n(\sigma_1^2+\sigma_2^2)}{2} & n=\mathrm{even}\\
    \frac{n(\sigma_1^2+\sigma_2^2)}{2}+\frac{\sigma_1^2-\sigma_2^2}{2} & n=\mathrm{odd}
    \end{cases}
\end{align}
\end{proof}

Given,
\begin{align}
    \lim_{n \to\infty}E\brak{\frac{S_n-a_n}{b_n}}=0
\end{align}
For all options, $a_n$ and $b_n$ are fixed numbers dependent only on $n$ and not a random variable. So, $E(a_n)=a_n$ and $E(b_n)=b_n$.
\begin{align}
    \implies \lim_{n \to\infty}\frac{E(S_n)-a_n}{b_n}=0\\
\end{align}
For all given options, 
\begin{align}
\lim_{n \to\infty}b_n=\infty 
\end{align}
So,
\begin{align}
\lim_{n \to\infty}E(S_n)-a_n=k, k\in\mathbb{R}, \mathrm{free\:of\:}n
\end{align}
Using corollary \eqref{eq2.0.4}, 
\begin{align}
\implies \frac{n(\mu_1+\mu_2)}{2}-a_n=k_1\\
\mathrm{And,\:}\frac{n(\mu_1+\mu_2)}{2}+\frac{\mu_1-\mu_2}{2}-a_n=k_2\\
\boxed{\implies a_n=\frac{n(\mu_1+\mu_2)}{2}+k_3\label{5}}
\end{align}
Given,
\begin{align}
    \lim_{n\to\infty}Var\brak{\frac{S_n-a_n}{b_n}}=1
\end{align}
Using lemma \eqref{3lemma},
\begin{align}
    \implies \lim_{n\to\infty}\frac{Var(S_n)}{b_n^2}=1\\
    \implies \frac{n(\sigma_1^2+\sigma_2^2)}{2b_n^2}=1,n=\mathrm{even}\\
    \mathrm{And,\:}\frac{n(\sigma_1^2+\sigma_2^2)}{2b_n^2}+\frac{\sigma_1^2-\sigma_2^2}{2b_n^2}=1,n=\mathrm{odd}\\
    \boxed{\implies b_n=\sqrt{n}\sqrt{\frac{\sigma_1^2+\sigma_2^2}{2}},n=\mathrm{even}\label{4}}\\
    \mathrm{And,\:}b_n=\sqrt{\frac{n(\sigma_1^2+\sigma_2^2)+(\sigma_1^2-\sigma_2^2)}{2}},n=\mathrm{odd}
\end{align}
Hence from equation \eqref{4} and \eqref{5}, the correct answer is option 1).
\subsection{Proof of Standard Normal Distribution: Classical Central Limit Theorem}
\begin{theorem}
If ${\textstyle X_{1},X_{2},\dots ,X_{n}}$ are a sequence of independent and identically distributed (i.i.d.) random variables drawn from a distribution with overall mean ${\textstyle \mu }$ and finite variance ${\textstyle \sigma ^{2}}$, and if ${\textstyle {\bar {X}}_{n}}$ is the sample average, i.e.,
\begin{align}
   \bar{X_n}=\frac{X_1+X_2+...+X_n}{n}, 
\end{align}
then the limiting form of the distribution,
\begin{align}
    {\textstyle Z=\lim _{n\to \infty }{\sqrt {n}}{\left({\frac {{\bar {X}}_{n}-\mu }{\sigma }}\right)}}\label{2.1.2}
\end{align}
is a standard normal distribution.
\end{theorem}
\begin{corollary}
$\frac{S_n-a_n}{b_n}$ converges to a standard normal distribution.
\end{corollary}
\begin{proof}
Let $Z_1=X_1+X_2$, $Z_2=X_3+X_4$, ... $Z_{n/2}=X_{n-1}+X_n$. As evident, $Z_1,Z_2,...,Z_{n/2}$ are independent and identically distributed random variables with common mean $(\mu_1+\mu_2)$ and common variance $(\sigma_1^2+\sigma_2^2)$.
By CLT, this implies
\begin{align}
    \lim_{\frac{n}{2}\to\infty}\sqrt{\frac{n}{2}}\frac{\frac{Z_1+Z_2+...+Z_{n/2}}{n/2}-(\mu_1+\mu_2)}{n/2}\label{2.1.3}
\end{align}
is a standard normal distribution.\par
By replacing $n/2$ by $n$ and by multiplying both numerator and denominator by $n$, the expression given in \eqref{2.1.3} comes out to be same as the expression $\frac{S_n-a_n}{b_n}$. Hence, the given expression converges in distribution to a standard normal distribution.
\end{proof}
\end{document}
