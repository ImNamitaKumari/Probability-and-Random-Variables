\documentclass[journal,12pt,twocolumn]{IEEEtran}
%
\usepackage{setspace}
\usepackage{gensymb}
\usepackage{caption}
%\doublespacing
\singlespacing
%\usepackage{graphicx}
%\usepackage{amssymb}
%\usepackage{relsize}
\usepackage[cmex10]{amsmath}
%\usepackage{amsthm}
%\interdisplaylinepenalty=2500
%\savesymbol{iint}
%\usepackage{txfonts}
%\restoresymbol{TXF}{iint}
%\usepackage{wasysym}
\usepackage{amsthm}
%\usepackage{iithtlc}
\usepackage{mathrsfs}
\usepackage{txfonts}
\usepackage{stfloats}
\usepackage{bm}
\usepackage{cite}
\usepackage{cases}
\usepackage{subfig}
\usepackage{xtab}
\usepackage{multirow}
%\usepackage{algorithm}
%\usepackage{algpseudocode}
\usepackage{booktabs}
\usepackage{enumitem}
\usepackage{mathtools}
\usepackage{tikz}
\usepackage{pgfplots}
\usepackage{circuitikz}
\usepackage{verbatim}
\usepackage{tfrupee}
\usepackage[breaklinks=true]{hyperref}
%\usepackage{stmaryrd}
\usepackage{tkz-euclide} % loads  TikZ and tkz-base
%\usetkzobj{all}
\usetikzlibrary{fit}
\usetikzlibrary{calc,math}
%\pgfdeclarelayer{background}
%\pgfsetlayers{background}
\usepackage{listings}
    \usepackage{color}                                            %%
    \usepackage{array}                                            %%
    \usepackage{longtable}                                        %%
    \usepackage{calc}                                             %%
    \usepackage{multirow}                                         %%
    \usepackage{hhline}                                           %%
    \usepackage{ifthen}                                           %%
  %optionally (for landscape tables embedded in another document): %%
    \usepackage{lscape}     
\usepackage{multicol}
\usepackage{chngcntr}
%\usepackage{enumerate}

%\usepackage{wasysym}
%\newcounter{MYtempeqncnt}
\DeclareMathOperator*{\Res}{Res}
%\renewcommand{\baselinestretch}{2}
\renewcommand\thesection{\arabic{section}}
\renewcommand\thesubsection{\thesection.\arabic{subsection}}
\renewcommand\thesubsubsection{\thesubsection.\arabic{subsubsection}}

\renewcommand\thesectiondis{\arabic{section}}
\renewcommand\thesubsectiondis{\thesectiondis.\arabic{subsection}}
\renewcommand\thesubsubsectiondis{\thesubsectiondis.\arabic{subsubsection}}

% correct bad hyphenation here
\hyphenation{op-tical net-works semi-conduc-tor}
\def\inputGnumericTable{}                                 %%

\lstset{
%language=C,
frame=single, 
breaklines=true,
columns=fullflexible
}
%\lstset{
%language=tex,
%frame=single, 
%breaklines=true
%}

\begin{document}
%


\newtheorem{theorem}{Theorem}[section]
\newtheorem{problem}{Problem}
\newtheorem{proposition}{Proposition}[section]
\newtheorem{lemma}{Lemma}[section]
\newtheorem{corollary}[theorem]{Corollary}
\newtheorem{example}{Example}[section]
\newtheorem{definition}[problem]{Definition}
%\newtheorem{thm}{Theorem}[section] 
%\newtheorem{defn}[thm]{Definition}
%\newtheorem{algorithm}{Algorithm}[section]
%\newtheorem{cor}{Corollary}
\newcommand{\BEQA}{\begin{eqnarray}}
\newcommand{\EEQA}{\end{eqnarray}}
\newcommand{\define}{\stackrel{\triangle}{=}}
\newcommand{\RomanNumeralCaps}[1]
    {\MakeUppercase{\romannumeral #1}}
\bibliographystyle{IEEEtran}
%\bibliographystyle{ieeetr}
\providecommand{\mbf}{\mathbf}
\providecommand{\pr}[1]{\ensuremath{\Pr\left(#1\right)}}
\providecommand{\qfunc}[1]{\ensuremath{Q\left(#1\right)}}
\providecommand{\sbrak}[1]{\ensuremath{{}\left[#1\right]}}
\providecommand{\lsbrak}[1]{\ensuremath{{}\left[#1\right.}}
\providecommand{\rsbrak}[1]{\ensuremath{{}\left.#1\right]}}
\providecommand{\brak}[1]{\ensuremath{\left(#1\right)}}
\providecommand{\lbrak}[1]{\ensuremath{\left(#1\right.}}
\providecommand{\rbrak}[1]{\ensuremath{\left.#1\right)}}
\providecommand{\cbrak}[1]{\ensuremath{\left\{#1\right\}}}
\providecommand{\lcbrak}[1]{\ensuremath{\left\{#1\right.}}
\providecommand{\rcbrak}[1]{\ensuremath{\left.#1\right\}}}
\theoremstyle{remark}
\newtheorem{rem}{Remark}
\newcommand{\sgn}{\mathop{\mathrm{sgn}}}
\providecommand{\res}[1]{\Res\displaylimits_{#1}} 
%\providecommand{\norm}[1]{\lVert#1\rVert}
\providecommand{\mtx}[1]{\mathbf{#1}}
\providecommand{\fourier}{\overset{\mathcal{F}}{ \rightleftharpoons}}
%\providecommand{\hilbert}{\overset{\mathcal{H}}{ \rightleftharpoons}}
\providecommand{\system}{\overset{\mathcal{H}}{ \longleftrightarrow}}
	%\newcommand{\solution}[2]{\textbf{Solution:}{#1}}
\newcommand{\solution}{\noindent \textbf{Solution: }}
\newcommand{\cosec}{\,\text{cosec}\,}
\providecommand{\dec}[2]{\ensuremath{\overset{#1}{\underset{#2}{\gtrless}}}}
\newcommand{\myvec}[1]{\ensuremath{\begin{pmatrix}#1\end{pmatrix}}}
\newcommand{\mydet}[1]{\ensuremath{\begin{vmatrix}#1\end{vmatrix}}}
\newcommand*{\permcomb}[4][0mu]{{{}^{#3}\mkern#1#2_{#4}}}
\newcommand*{\perm}[1][-3mu]{\permcomb[#1]{P}}
\newcommand*{\comb}[1][-1mu]{\permcomb[#1]{C}}
%\newcommand*{\perm}[2]{{}^{#1}\!P_{#2}}%
%\newcommand*{\comb}[2]{{}^{#1}C_{#2}}%
%\numberwithin{equation}{section}
\numberwithin{equation}{subsection}
%\numberwithin{problem}{section}
%\numberwithin{definition}{section}
\makeatletter
\@addtoreset{figure}{problem}
\makeatother
\let\StandardTheFigure\thefigure
\let\vec\mathbf
%\renewcommand{\thefigure}{\theproblem.\arabic{figure}}
\renewcommand{\thefigure}{\theproblem}
%\setlist[enumerate,1]{before=\renewcommand\theequation{\theenumi.\arabic{equation}}
%\counterwithin{equation}{enumi}
%\renewcommand{\theequation}{\arabic{subsection}.\arabic{equation}}
\def\putbox#1#2#3{\makebox[0in][l]{\makebox[#1][l]{}\raisebox{\baselineskip}[0in][0in]{\raisebox{#2}[0in][0in]{#3}}}}
     \def\rightbox#1{\makebox[0in][r]{#1}}
     \def\centbox#1{\makebox[0in]{#1}}
     \def\topbox#1{\raisebox{-\baselineskip}[0in][0in]{#1}}
     \def\midbox#1{\raisebox{-0.5\baselineskip}[0in][0in]{#1}}
\vspace{3cm}
\title{Assignment 5}
\author{Namita Kumari - CS20BTECH11034}
\maketitle
\newpage
\bigskip
Download all latex-tikz codes from 
%
\begin{lstlisting}
https://github.com/ImNamitaKumari/Probability-and-Random-Variables/blob/main/Assignment5/Assignment5.tex
\end{lstlisting}
\section{CSIR UGC NET - June 2013 Q. 75}
Let $X$ be a non-negative integer valued random variable with probabilty mass function $f(x)$ satisfying $(x+1)f(x+1)=(\alpha+\beta x)f(x)$, x=0,1,2, ...; $\beta\neq 1$. You may assume that $E(X)$ and $Var(X)$ exist. Then which of the following statements are true?
\begin{enumerate}
\item $E(X)=\frac{\alpha}{1-\beta}$
\item $E(X)=\frac{\alpha^2}{(1-\beta)(1+\alpha)}$
\item $Var(X)=\frac{\alpha^2}{(1-\beta)^2}$
\item $Var(X)=\frac{\alpha}{(1-\beta)^2}$
\end{enumerate}
\section{Solution}

\begin{definition}
\label{1}
    \begin{align}
        A(n)&=\sum_{x=0}^nf(x)\\
        B(n)&=\sum_{x=0}^nxf(x)\\
        C(n)&=\sum_{x=0}^nx^2f(x)
    \end{align}
\end{definition}

\begin{definition}
\label{2}
Sum of probabilities is equal to 1.
\begin{align}
\lim_{n\to\infty}A(n)&=1\\
\lim_{n\to\infty}B(n)&=E(X)\\
\lim_{n\to\infty}C(n)-(E(X))^2&=Var(X)
\end{align}
\end{definition}

\begin{lemma}
\label{2.1}
\begin{align}
\lim_{n\to\infty}nf(n)=0 =\lim_{n\to\infty}n^2f(n)
\end{align}
\end{lemma}
\begin{proof}
As given in question, $E(X)$ and $Var(X)$ are finite.
\begin{align}
    E(X)=\sum xf(x)\\
    Var(X)=\sum x^2f(x)-(E(X))^2
\end{align}
Hence at infinity, the contribution of $nf(n)$ and $n^2f(n)$ has to be zero for mean and variance to be finite.
\end{proof}

\begin{lemma}
\label{2.2}
\begin{align}
    B(n)=\frac{\alpha A(n)-(n+1)f(n+1)}{1-\beta}
\end{align}
\end{lemma}
\begin{proof}
\begin{align}
    (x+1)f(x+1)=(\alpha+\beta x)f(x)\label{2.0.6}\\
    \implies \alpha x = (x+1)f(x+1) - \beta xf(x)
\end{align}
\begin{align}
    \implies\alpha f(0) &= f(1)-0\label{2.0.8}\\
    \alpha f(1) &= 2f(2) - \beta f(1)\\
    \alpha f(2) &= 3f(3) - 2\beta f(2)\\
    &\vdots\\
    \alpha f(n) &= (n+1)f(n+1) - n\beta f(n)\label{2.0.12}
\end{align}
Adding equations from \eqref{2.0.8} to \eqref{2.0.12},
\begin{align}
  &\alpha\sum_{x=0}^nf(x)=(n+1)f(n+1)+(1-\beta)\sum_{x=0}^nxf(x)
\end{align}
Using definition \eqref{1},
\begin{align}
    &\implies B(n)=\frac{\alpha A(n)-(n+1)f(n+1)}{1-\beta}
\end{align}
\end{proof}

\begin{lemma}
\label{2.3}
    \begin{align}
        C(n)=\frac{(\alpha+\beta)B(n)+\alpha A(n)-(n+1)^2f(n+1)}{1-\beta}
    \end{align}
\end{lemma}
\begin{proof}
    Multiplying equation \eqref{2.0.6} with $(x+1)$ on both sides and rearranging,
    \begin{align}
        (x+1)^2f(x+1)-\beta x^2f(x)=(\alpha+\beta)xf(x)+\alpha f(x)
    \end{align}
    \begin{align}
        \implies 1^2f(1)-0&=0+\alpha f(0)\label{2.0.17}\\
        2^2f(2)-1^2\beta f(1)&=(\alpha+\beta)f(1)+\alpha f(1)\\
        3^2f(3)-2^2\beta f(2)&=2(\alpha+\beta)f(2)+\alpha f(2)\\
        &\vdots\\
        (n+1)^2f(n+1)&-n^2\beta f(n)=n(\alpha+\beta)f(n)+\alpha f(n)\label{2.0.22}
    \end{align}
    Adding equations from \eqref{2.0.17} to \eqref{2.0.22},
    \begin{multline}
        (n+1)^2f(n+1)+(1-\beta)\sum_{x=0}^nx^2f(x)=\alpha\sum_{x=0}^nf(x)\\+(\alpha+\beta)\sum_{x=0}^nxf(x)\label{2.0.25}
    \end{multline}
    Using definition \eqref{1},
    \begin{align}
        \implies C(n)=\frac{(\alpha+\beta)B(n)+\alpha A(n)-(n+1)^2f(n+1)}{1-\beta}
    \end{align}
\end{proof}

Using definition \eqref{2} and lemma \eqref{2.2}, 
\begin{align}
    \lim_{n\to\infty}B(n)&=\lim_{n\to\infty}\frac{\alpha A(n)-(n+1)f(n+1)}{1-\beta}\\
    &=\lim_{n\to\infty}\frac{\alpha-(n+1)f(n+1)}{1-\beta}
\end{align}
Using lemma \eqref{2.1},
\begin{align}
    \boxed{\implies E(X)=\frac{\alpha}{1-\beta}}\label{2.0.31}
\end{align}

Using definition \eqref{2} and lemma \eqref{2.3},
\begin{align}
    \lim_{n\to\infty}C(n)=\frac{(\alpha+\beta)E(X)+\alpha-(n+1)^2f(n+1)}{1-\beta}
\end{align}
Using lemma \eqref{2.1},
\begin{align}
    &\lim_{n\to\infty}C(n)=\frac{(\alpha+\beta)E(X)+\alpha}{1-\beta}\\
    &\implies Var(X)=\frac{(\alpha+\beta)E(X)+\alpha}{1-\beta}-(E(X))^2
\end{align}
Using equation \eqref{2.0.31},
\begin{align}
    Var(X)=\frac{\frac{\alpha(\alpha+\beta)}{1-\beta}+\alpha}{1-\beta}-\frac{\alpha^2}{(1-\beta)^2}
\end{align}
\begin{align}
    \boxed{\implies Var(X)=\frac{\alpha}{(1-\beta)^2}}
\end{align}
Hence, the correct options are 1) and 4).
\end{document}
